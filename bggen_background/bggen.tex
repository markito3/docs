\documentclass{article}
\begin{document}
The probability of scattering $p = \sigma/a_{\rm int}$, where $\sigma$ is the scattering cross section and $a_{\rm int}$ is the ``area of interaction'' (which will cancel in the end). The number of scattering events $N=pN_bN_t$ where $N_b$ is the number of beam particles on target and $N_t$ is the number of target particles in the area of interaction. So
$$
N = {\sigma N_b N_t \over a_{\rm int}}
$$
  $N_t = V\rho_n$ where $V$ is the volume of the target and $\rho_n$ is the number density of target particles. $V=a_{\rm int}l$ where $l$ is the length of the target. So $N_t = a_{\rm int}l\rho_n$. $\rho_n = \rho/m_t$ where $\rho$ is the mass density of the target and $m_t$ is the mass of a single target particle. So $N_t = a_{\rm int}l\rho/m_t$. If the target is a nucleus, $m_t = A/N_A$ where $A$ is the atomic weight (in grams per mole usually) of the target particle and $N_A$ is Avogadro's number. So $N_t = a_{\rm int}l\rho N_A/A$ and 
$$
N = {\sigma N_b l \rho N_A \over A}
$$

$$
N_b = Rt
$$
where $R$ is the beam photon time rate and $t$ is the time of running.
  
%  $\sigma$ & 124~$\mu$b & original GlueX proposal[?]
  
\end{document}
