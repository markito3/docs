\chapter{Elements of Group Theory}

\section{Correspondences and transformations.}

{\bf p.\ 3}: A restatement with a cyclic permutation:
$$
\genfrac(){0pt}0{1234}{2341}
$$
means that the object in box 1 goes to box 2, the object in box 2 goes
to box 3, {\it etc}. The numbers refer to the boxes and not the
objects. In this notation the order of the columns is irrelevant,
i.e.,
$$
\genfrac(){0pt}0{1324}{2431}
$$
is the same permutation.  If the objects were identified as A, B, C,
and D and start out in boxes 1, 2, 3, and 4 respectively,
$$
\genfrac{}{}{0pt}0{\tt 1234}{\tt ABCD}
$$
after application of the permutaion the box/object arrangement ends up as
$$
\genfrac{}{}{0pt}0{\tt 1234}{\tt DABC}
$$

p.\ 4: The degree of a permutation is simply the number of objects
being permuted.

\section{Groups. Definitions and examples.}

By a group of tranformations, $G$, we mean an aggregate of
transformations of a given point set which
\begin{enumerate}
\item contains the identity transformation;
\item for every transformation $M$, also contains its inverse $M^{-1}$;
\item if it includes $M$ and $M^\prime$, also includes ther composite $MM^\prime$.
\end{enumerate}

[Rather than transformation, could use elements. Rather than
  composite, could use result of the group operation.]

{\bf p.\ 7, Order of a Group}. The number of elements in a group is
called the {\it order} of the group.

{\bf Structure of a Group}. The structure of an abstract group is
determined by stating the result of the ``multiplication'' of every
ordered pair of elements, either by enumeration or by stating a
functional law, without any particularization of the nature of the
elements.

{\bf p.\ 8, Order of a group element}. If all powers of an element $a$
are distinct, then $a$ is of {\it infinite order}. If $n$ is the
smallest integer such that $a^n = e$ then $n$ is the order of the
element $a$.

{\bf Example of the order of a group element}. The cyclic group of order 4 is

\begin{equation*}
\begin{split}
e = (1234) \\
a = (2341) \\
b = a^2 = (3412) \\
c = a^3 = (4123)
\end{split}
\end{equation*}

The order of $e$ is 1 ($e^1 = e$). The order of $a$ is 4 ($a^4 = e$). The order of $b$ is 2 ($b^2 = e$). The order of $c$ is 4 since ($c^4 = e$).

{\bf p.\ 11}: The {\it four group} (a.k.a $V$) is

\begin{equation}
\begin{split}
  eabc \\
  aecb \\
  bcea \\
  cbae
\end{split}
\end{equation}
with the properties that $a^2 = b^2 = c^2 = e$, $ab = c$, $ac = b$,
and $bc = a$. Note that the four group has three elements (a, b, and
c) of order 2, and one element (e) of order 1. So none of the elements
of this group of order 4 is of order 4.

{\bf p.\ 12}. The permutations of degree $n$ form a group called the
{\it symmetric group} of degree $n$, denoted by $S_n$.

{\bf Note on transposition notation} (p.\ 13). The order of the
transpositions is important. They are applied right-to-left. As an example
(13)(12) should be intepreted as a transformation where
first (12) is applied followed by (13). Continuing with the example:
$$
\genfrac(){0pt}0{123}{231}
$$
is a cyclic permutation. If we start with box/objects
$$
\genfrac{}{}{0pt}0{\tt 123}{\tt ABC}
$$
then application of this permutation gives
$$
\genfrac{}{}{0pt}0{\tt 123}{\tt CAB}
$$
One way of expressing this as a sequence of transpositions is to apply the
transposition (13), which gives
$$
\genfrac{}{}{0pt}0{\tt 123}{\tt CBA},
$$
followed by (23), which gives
$$
\genfrac{}{}{0pt}0{\tt 123}{\tt CAB},
$$
and is also equivalant to our example. Going back to the example, applying the
transposition (12), which gives
$$
\genfrac{}{}{0pt}0{\tt 123}{\tt BAC},
$$
followed by (13), which gives
$$
\genfrac{}{}{0pt}0{\tt 123}{\tt CAB}.
$$
This is not the same as (13) followed by (12), which gives
$$
\genfrac{}{}{0pt}0{\tt 123}{\tt BCA}.
$$

A nice page with examples of cycles:
https://math.stackexchange.com/questions/319979/how-to-write-permutations-as-product-of-disjoint-cycles-and-transpositions
.

{\bf Uniqueness of representation of a cycle by transpositions}
(p.\ 13). They are not unique. The number of transpositions can vary
and even with a given number of transpositions, the representation is
not unique. Example: (123) = (13)(12) = (12)(23). The first
representation is the one called out in the text.

{\bf p.\ 14}: Call (a...xb...y) ``the cycle.'' Ignoring other possible
independent cycles, we start with
$$
\genfrac{}{}{0pt}0{\tt 123456}{\tt ABCDEF}\quad,
$$
thus assuming $a=1$, $x=3$, $b=4$, and $y=6$. Applying the cycle we get
$$
\genfrac{}{}{0pt}0{\tt 123456}{\tt FABCDE}\quad.
$$
Now apply (ab), aka (14). We get
$$
\genfrac{}{}{0pt}0{\tt 123456}{\tt CABFDE}\quad.
$$
so this looks like (123)(456).

More generally, the cycle moves all objects to the next box, and in
particular the object in box $x$ moves to box $b$ and the object in
box $y$ moves to box $a$. So subsequent application of (ab) puts the
object that started in box $x$ into box $a$ and the object that
started in box $y$ into box $b$. The net effect is that objects in
boxes $a,...,x$ cycle among themselves. Likewise for the ojbect in
boxes $b,...,y$. I.e.,
$$
(ab)(a...xb...y) = (a...x)(b...y).
$$

For an example of a cycle broken into transpositions see
example\_cycle\_to\_transpositions.pdf.

{\bf Decrement of a permutation} (p.\ 14) The difference of these
numbers, $7-3 = 4$, is called the decrement of the permutation. (In
this case the permutation is $(123)(45)(67)$ where 7 is the number of
permuted symbols and 3 is the number of independent cycles in cycle
notation, which in the general case includes single-member cycles.)

{\bf Even and odd permutations} (p.\ 14) Permutations with even (odd)
decrement are said to be even (odd) permutations.

{\bf Proof} (p.\ 14) The reader should supply the proof that if the
decrement of a permutation is even (odd), its resolution into a product
of transpositions will have an even (odd) number of factors.
\begin{itemize}
\item A cycle on $n$ elements is a product ot $n-1$ transpositions.
  \begin{itemize}
  \item on (1), 0 transpositions
  \item on (12), (12) or 1 transposition
  \item on (123), (13)(12) or 2 transpositions
  \end{itemize}
\item Each cycle of an odd(even) number of members contributes an
  even(odd) number of transpositions
\item If the number of independent cycles has $m$ even-membered cycles
  and $n$ odd-membered cycles then the number of transpositions needed
  is even or odd as $m$ is even or odd because the even-membered
  cycles contribution an odd number of transpositions.
\item If the total number oe elements $N$ is even (odd), then the
  number of odd-membered cycles ($n$) must be even (odd).
\item By definition, the decrement $d = N-(m+n)$
\item Assume $N$ is odd. Then $n$ is odd. If $m$ is even, $d$ is
  even. If $m$ is odd, $d$ is odd. So $d$ is even or odd as is the
  number of transpositions.
\item Assume $N$ is even. Then $n$ is even. The same reasoning as
  above gives $d$ even or odd as the number of transpositions is even
  or odd for this case as well.
\item $d$ is even or odd as the number of transpositions is even or
  odd for all cases. QED.
\end{itemize}

\section{Subgroups. Cayley's theorem}

A subgroup is a subset of elements of a group that is a group
itself. Two trivial subgroups: the identity element and the entire
group. These are improper subgroups, as opposed to proper subgroups.

The subgroup must use same group operation as the group.

p. 15: An improper subgroup is the either the identity element alone
or the whole group. Other subgroups are proper.

p. 15: Since the product of any pair of elements of $\mathcal H$ are
in $\mathcal H$ and each element of $\mathcal H$ contains its inverse,
for any element $a$ in $\mathcal H$, $aa^{-1} = e$ says that $e$ is an
element of $\mathcal H$.

p. 16: Notes on Cayley's Theorem: The symmetric group of degree $n$
has order $n!$. Cayley's Theorem refers to groups of order $n$ being
subgroups of the symmetric group of degree $n$ (not order $n$).

p.\ 17: Stated in the text: $ba_i = ba_j$ means $a_i = a_j$. The
result of both sides in the premise must be members of the group. The
statement is that they cannot result in the same member if $a_i$ and
$a_j$ are distinct members of the group. QUESTION: why? ANSWER: If
$ba_i = ba_j$ then $b^{-1}ba_i = b^{-1}ba_j$. Then $ea_i = ea_j$ and
so $a_i = a_j$. CONVOLUTED ANSWER:
\href{https://planetmath.org/uniquenessofinverseforgroups}{Inverses
  are unique}. Assume that $ba_i = ba_j$ and $a_i \neq a_j$. Then
$a^{-1}_i b^{-1} = a^{-1}_j b^{-1}$, and $a^{-1}_i b^{-1} b = a^{-1}_j
b^{-1} b$. So $a^{-1}_i = a^{-1}_j$ and since inverses are unique $a_i
= a_j$ which contradicts the assumption. So if $a_i \neq a_j$ then
$ba_i \neq ba_j$.

p. 19: A regular permutation is one for which no elements remain in
their original order. No definition of a regular subgroup is given. My
take: a regular subgroup is one where all elements are regular
permutations excluding the identity.

p. 19: Alternate statement of a property of certain
subgroups:\hfil\break Groups with only regular permutations (other
than the identity) have the following property. For any given box,
and for a given element (i.e., permutation) of the group, the image
box under the permutation is distinct from that of all other elements
of the group.

p. 20: Group $G$ is {\it cyclic} if there exists $a$, a member of $G$,
such that the cyclic subgroup generated by $a$ ($a,a^2,a^3,...$),
equals all of $g$

p.21: With reference to a proper subgroup $H$ of order $m$, of a
finite group $G$ of order $n$, where $n/m = h$, $h$ is called the
index of the subgroup. The set of members of the $G$ formed by
multiplying by $a$ on the left of each memeber of $H$, where $a$ is a
member of $G$ and not a member of $H$, is called a left coset of
$H$. Multiplying on the right gives a right coset. Note that cosets
cannot be subgroups of $G$ since they do not contain the identity.

\section{Cosets. Lagrange's theorem.}

{\bf Lagrange's Theorem}. The order of a subgroup of a finite group is
a divisor of the order of the group.

{\bf p.\ 22}: If the order of an element $a$ of a group $G$ is $n$
then $a^n = e$. By definition of the order of an element,
$a,a^2,...,a^n$ are all distinct. So these powers of $a$ form a group
of order $n$. I.e., if $a$ is an element of order $n$, it is a member
of a subgroup of $G$ of order $n$.

The cyclic groups are
\href{https://en.wikipedia.org/wiki/Cyclic_group}{defined on
  Wikipedia}. From that article:

\begin{quote}
Every finite cyclic group of order $n$ is isomorphic to the additive
group of $Z/nZ$, the integers modulo n.
\end{quote}

% group of order 6 discussion

Lagrange's theorem can be used for finding the possible structures of
groups of a given order. As an example we find all structures of
order 6. Since the order of the group is six the order of each of its
elements is a divisor of 6 i.e., 1, 2, 3 or 6.
\begin{quote}
If an element $a$ is of order $n$ then the group elements
$a,a^2,...,a^n$ are distinct and form a subgroup of order $n$. By
Lagrange's theorem, $n$ must be a divisor of 6.
\end{quote}
If the group contains an element
$a$ of order 6 then the group is the cyclic group $a,a^2,...,a^6 = e$.
\begin{quote}
This subgroup has the same structure as the cyclic group, so it is in
fact the cyclic group.
\end{quote}
To find other possible structures, we suppose that the group
contains no element of order 6, but has an element $a$ of order
3. Thus the group contains the subgroup $a,a^2,a^3 = e$.
\begin{quote}
  $a$ cannot be $e$ since $a^2 = e$ and $a$ is a member of a group of
  order 1. $a^2$ cannot be $e$ otherwise $aa^2 = a^3$ would be equal
  to $a$ and thus $a$ and $a^3$ would not be distinct.
\end{quote}
  If the group also contains another element, $b$, then it contains
  the six distinct elements $e,a,a^2,b,ba,ba^2$.
\begin{quote}
  $b$ is distinct from $e,a,a^2$ by assumption. $ba$ is distinct from
  $b$ since $a$ is not the identity. It is also distinct from $e$,
  $a$, and $a^2$ since if we apply $a^{-1}$ on the right to any of the
  three we get $b$ and if we apply $a^{-1}$ on the right to any of
  $e$, $a$, or $a^2$ we would get one of $e$, $a$, or $a^2$ which are
  distinct from $b$, a contradiction. A similar argument shows $ba^2$
  being distinct from the other five elements. Also these six elements
  exhaust the group since we assume a group of order six.
\end{quote}
The element
$b$ has order two or three.
\begin{quote}
  It cannot be of order one since it is distinct from the identity by
  assumption. It cannot be order six since we have assumed that there
  is no element of order 6.
\end{quote}
If the order of $b$ is 3, that is, $b^3 = e$,
the element $b^2$ must be one of the six elements listed.
We cannot have $b^2 = e$ (since we assumed that $b$ is of
order 3), and $b^2 = b, ba$, or $ba^2$ implies $b
= e, a$, or $a^2$, respectively, which contradicts our
assumption that $b$ is distinct from these elements. Furthermore
$b^2 = a$ implies $ba = e$, and $b^2 = a^2$ implies $ba = e$,
both of which contradict our assumptions. Thus the order of $b$
cannot be 3 and we must have $b^2 = e$. The product $ab$ cannot
be $e, a, a^2$, or $b$. If $ab = ba$ then
\begin{multline}
(ab)^2 = (ab)(ab) = (ab)ba = ab^2a = a^2, \\
(ab)^3 = a^2(ab) = b; (ab)^4 = a, (ab)^5 = ba^2, (ab)^6 = e,
\end{multline}
and therefore the group would contain an element $ab$ of order 6
contrary to assumption. (This can be done more briefly: $a$ is of order
2, $b$ of order 3, and $ab = ba$, that is, $a$ and $b$ commute. Taking powers
we see that the order of $ab$ is the least common multiple of the
orders of $a$ and $b$.) We are now left with
$$
b^2 = e, ab = ba^2;
$$
$ab = ba^2$ implies $(ab)^2 = (ab)ba^2 = e$.

This last assumption leads to no contradictions.

[Abandoning explanation of how Lagrange's theorem tells us the
  structure of groups of order 6.]

\section{Conjugate classes.}

Q: For any given group element $a$, does the group contain $a^{-1}$?

A: Yes. It is property (2) of the definition of a group, p.\ 6.

{\bf Note on uniqueness of transformed group members}: Assume $a$,
$b$, and $c$ are members of a group and that $aba^{-1} =
aca^{-1}$. Applying $a^{-1}$ on the left and $a$ on the right of both
sides gives $b=c$. So transformations of any two distinct members
($b$, $c$) by the same member ($a$) must be distinct, since no
transformation of distinct members can map to the same member ($b$
must be the same as $c$ if they transform to the same member).


{\bf Definition}: A conjugate class is a subset of a group such that
all members are conjugate to one another.
\begin{itemize}
  \item The transforming group element may or may not be a member of
    the class per this definition.
  \item For abelian grounps, each element of the group forms a class by itself.
  \item On p.\ 23, $\equiv$ is read ``is conjugate to.''
\end{itemize}

Comment, top of p.\ 24, ``all elements in the same class have the same
order'': If the order of $a$ is $h$, then by definition no lower power
of $h$ gives $e$. So no lower power of $ubu^{-1}$ gives $e$ either. So
no lower power of $b$ gives $e$. So $b$ is of order $h$ as well.

Example, p.\ 24: $a$ is a reflection in plane $P$, $c$ is the rotation
about some axis, $x$ is a point in $P$. $a$ leaves $x$ invariant,
i.e., $ax = x$.
$$
(cac^{-1})(cx) = cax = cx
$$ because of the properties of inverses and the invariance of $x$
when transformed by $a$. So $cx$ is invariant with respect to the
transformation $cac^{-1}$. $cx$ is in the plane resulting from
applying the rotation $c$ to the plane $P$, i.e., it is in the rotated
$P$. The text claims that $cac^{-1}$ is therefore the reflection in
the rotated plane $P$. Don't see why that has to be true based on a
conclusion about points $x$ in the plane $P$.

{\bf The procedure works equally well...} (p.\ 25). The ``procedure''
is to apply $b$ lexically to $a$. In the example, the first move in $b$
is to go from 2 to 4, in $a$, we replace 2 by 4, lexically. That this
``works'' is not demonstrated in the text. Demonstrated explicitly
for example in text. See conjugate\_cyclic.pdf.

For solution to problem on p.\ 25, see conjugate\_classes\_S\_5.pdf.

{\bf Definition: cycle structure} (p.\ 25): A permutation which when
resolved into independent cycles has $\nu_1$ 1-cycles, $\nu_2$
2-cycles,..., $\nu_n$ $n$-cycles is said to have the cycle structure
$(1^{\nu_1},2^{\nu_2},...,n^{\nu_n})$, or briefly $\nu$.

Eq.\ (1-23) (p.\ 25) refers to a single permutation, i.e., a single element of $S_n$.

For solution to problem on p.\ 27, see conjugate\_class\_problem\_p27.pdf.

For solution to the other problem on p.\ 27, see class\_count.pdf. Note
that each class has a cycle structure denoted by $(\nu)$.

{\bf Missing verbiage} (p.\ 27) Just as (13)(24) is the transform of
(12)(34) by (23), (14)(23) is the transform of (13)(24) by (34), using
the language on p.\ 23.

{\bf Four group discussion} (p.\ 27) Do not understand ``permutations
are not in the same class in $V$.'' That implies permutations are in
$V$, but do not see how permutations would map onto $V$. Unsolved
mystery.

For the solution to the problem at the top of p.\ 28, see
problem\_solution\_p28.pdf

\section{Invariant subgroups. Factor group. Homomorphism}

Starting from a subgroup $\mathcal H$ of group $G$ and any member of
the group $a$, the set of elements $aha^{-1}$, where $h$ runs over all
members of the subgroup $\mathcal H$, is a subgroup itself and is
called a conjugate subgroup of $\mathcal H$ in $G$. If for all $a$,
$a{\mathcal H} a^{-1}$ = $\mathcal H$, we say that $\mathcal H$ is an
invariant subgroup in $G$.

For verification that $a\mathcal Ha^{-1}$ is a subgroup of G, see
subgroup\_verify.pdf.

{\bf Note on ``given an element $h_1$...$ah_1a^{-1} = h_2$}'' (p.\ 28) This is true for each member of $\mathcal H$. And if we cycle
through all members of of $\mathcal H$ (each ``$h_1$'') and collect
all of the resulting ``$h_2$'''s we recover $\mathcal H$.

{\bf Explication of statement about complete classes} (p.\ 28) A class
is the set of elements such that when any one of them is transformed
by any member of the group, the result remains in the set. The
definition of an invariant subgroup says that any member of the
subgroup, when transformed by any member of the group, results in
another member of the subgroup.  Not only that but when each member of
the subgroup is transformed by each member of the group in turn, the
resulting set of tranformed members recovers the original
subgroup. Assume that the invariant subgroup contains a member of a
class. Then transformation of that member by any group member results
in a member of both the class and the subgroup. Assume that there is
another member of the class which is not a member of the
subgroup. Transformation of this member will result in another member
of the class, but also in another non-member of the subgroup, because
the reverse transformation yields the original non-member and if the
transformed member were a member of the subgroup, any transformation
would yield a member of the subgroup, a contradiction. This would mean
that the class is consists of at least two disjoint sets, members that
are in the subgroup and those that are not. It also means that members
of one set cannot be transformed to members of the other set which
contradicts the assumption that they are all members of the same
class. The class need not exhaust the subgroup. The subgroup need only
contain all of the members of the class. The identity is an example of
an element which need not be a member of the class.

``All the subgroups of an abelian are clearly invariant'' (p.\ 28) since $aha^{-1} = aa^{-1}h = h$, i.e., all transformations of member of an abelian group recover the original member.

Solution to problem (1), p.\ 29 is shown in problem1\_p29.pdf. 
\href{https://math.stackexchange.com/questions/84632/subgroup-of-index-2-is-normal}{Solution to problem (2)}, p.\ 29 is on Stack Exchange. Solution to problem (3) is shown in problem3\_p29.pdf.

...isomorphism for group...a one-to-one correspondences was set up
between elements $a$ of group $G$ and elements $a^\prime$ in a group
$G^\prime$, such that $(ab)^\prime = a^\prime b^\prime$. By a
homomorphic mapping of $G$ on $G^\prime$ we mean a similar
correspondence which preserves prducts, but now several elements of
$G$ may have the same image in $G^\prime$. [Example: cyclic group of
  order four mapped into cyclic group of order 2.]

\section{Direct Products}

A group $G$ is said to be a direct product of its subgroups ${\mathcal
  H_{\rm 1},...,H_{\rm n}}$ if:

\begin{enumerate}
\item The elements of different subgroups commute.
\item Every element of G is expressible in one and only one way as
  $g = h_1...h_n$, where $h_i$ is in ${\mathcal H}_i$, for $i = 1,...,n$
\end{enumerate}

{\bf Note on ``the subgroups $H_i$ have only the identity in common.''}
Assume that two subgroups $H_1$ and $H_2$ have an element in common
that is not the identity, $h_{12}$. Then h12 $h_{12}$ itself can be
expressed as the product of the identity from each subgroup except for
$h_{12}$ from $H_1$, or as the identity from each subgroup except for
$h_{12}$ from $H_2$. This contradicts the uniqueness of expression of
h1 $h_{12}$2 as a product of elements, one from each of the subgroups.
