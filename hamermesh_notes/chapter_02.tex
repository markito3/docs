\chapter{Symmetry Groups}

{\bf General Note} A ``symmetry'' is a tranformation that leaves the
physical system in a state equivalent to the original. A ``symmetry
group'' is a group of such transformations each of which may or may
not be symmetries of a particular physical system[?].

{\bf Point groups} (p.\ 33) comprise of only reflections in a plane and
rotations about a fixed axis and do not include translations. This
implies that all rotations and reflections leave at least one point
fixed. For example, rotations about non-intersecting axes would result
in a translation. As another example, reflections through
non-intersecting planes would also result in a translation.

{\bf Comments on ``Thus for odd $n$, if the body has the symmetry
  $S_n$, it also has the symmetries $\sigma_h$ and $C_n$ as
  independent symmetry elements.''} (p.\ 33) $S_n$ being a symmetry,
means operation by $S_n$ returns the object to its original
configuration. Thus $(S_n)^n$ is a symmetry. In addition, if $n$ is
odd then $(S_n)^n$ is $\sigma_h$. So $\sigma_h$ is a symmetry. If
$S_n$ and $\sigma_h$ are symmetries, then $C_n$ must also be a
symmetry, since $C_n \sigma_h$ (i.e., $S_n$) has to return the object
to its original confuguration. Take as an example a three-fold
arrangement above the reflection plane and one the same distance below
the plane where one is rotated by 60 degrees from the other.  This
does not have symmetry $S_3$. It does have symmetry $S_6$. It also has
symmetry $C_3$. It does not have symmetry $\sigma$.  Now take an
arrangement like the previous, but without the 60 degree
rotataion. This does have symmetry under $S_3$, it does have symmetry
under $\sigma$, and it does have symmetry under $C_3$. But note that
$n$ is odd ($n = 3$).

{\bf Comment on ``All three elements $I$, $\sigma_h$, and $C_2$
  commute.} (p.\ 37) $\sigma_h$ and $C_2$ commute (as noted on the
previous page). Remember that $I = S_2$. Since $S_2 = \sigma_h C_2$ by
definition, $I$ commutes with $\sigma_h$ and $C_2$.

{\bf Comments on ``If any two of these elements belong to the symmetry
  group, then so does the third.''} (p.\ 37) Eqs.\ (2.6) and (2.6a)
show that each of the elements can be expressed as a combination of
the other two. Also all three commute with one another. So if two are
in the group, so must the third since we can produce the third as the
product of the two.

{\bf Comment on Fig.\ 2-6.} (p.\ 37) The construction does not depend
on $P^\prime$ lying on the bisection of the angle $\phi$, as one might
infer from the drawing. In other words, point $P$ could be anywhere on
the plane, including in the interior of the angle $\phi$.

{\bf Explanation of ``...if the $X$- and $Y$-axes are 2-fold rotation
  axes, then so it the $Z$-axis.''} (p.\ 38). $n$-fold rotation axis
means that the object is symmetric under rotation by $2\pi/n$ about
that axis (defined on p.\ 33).  A rotation of $\pi$ about $X$ leaves
the object in coincidence with itself. A following rotation of $\pi$
about $Y$ also leaves the object in coincidence with itself. But the
result of these two operations is a rotation of $\pi$ about $Z$ (since
$X$ and $Y$ subtend an angle of $\pi/2$). Therefore, $Z$ is a 2-fold
rotation axis.

{\bf Explanation of ``...if the line $OP$ is an $n$-fold axis, then we
  must also have $n$-fold axes along $OP^\prime...$''} (p.\ 38) If OP
is a $n$-fold rotation axis then if a $2\pi/n$ rotation about $OP^\prime$
does not bring the object into coincidence with itself, that means
that the distribution of points with respect to $OP^\prime$ is
different than that with respect to $OP$. That contradicts the
assumption that the vertical axis is a 3-fold symmetry axis.

{\bf ...all belong to the same class...} (p.\ 38) A $2\pi/n$ rotation about
$OP^\prime$ can be obtained by a $2\pi/3$ rotation bringing $OP^\prime$ into
coincidence with $OP$, followed by a rotation of $2\pi/n$ about $OP$,
followed by the reverse rotation of $2\pi/3$ back to $OP^\prime$. So the
rotation about $OP$ by $2\pi/n$ is conjugate to that about $OP^\prime$. In a
similar manner all $n$-fold rotations about $OP$, $OP^\prime$, and
$OP^{\prime\prime}$ are conjugate to one another. They thus form a
conjugate class.

{\bf Question: What is an example of an object that has multiple equivalent axes and rotation-reflection symmetry?} (p.\ 38)
A molecule with atoms at (2,0,1), (0,2,1), (-2,0,1), (0,-2,1), (2,0,-1), (0,2,-1), (-2,0,-1), and (0,-2,-1) has four equivalent axes, the positive $x$-axis, the negative $x$-axis, the positive $y$-axis, and the negative $y$-axis. It has reflection symmetry about the $x$-$y$ plane, the $x$-$z$ plane, and the $y$-$z$ plane.
