\chapter{Symmetry Groups}

{\bf Comments on ``If any two of these elements belong to the symmetry
  group, then so does the third.''} (p.\ 37) Eqs.\ (2.6) and (2.6a)
show that each of the elements can be expressed as a combination of
the other two. Also all three commute with one another. So if two are
in the group, so must the third since we can produce the third as the
multiplication of the two.

{\bf Explanation of ``...if the $X$- and $Y$-axes are 2-fold rotation
  axes, then so it the $Z$-axis.''} (p.\ 38). $n$-fold rotation axis
means that the object is symmetric under rotation by $2\pi/n$ about
that axis (defined on p.\ 33).  A rotation of $\pi$ about $X$ leaves
the object in coincidence with itself. A following rotation of $\pi$
about $Y$ also leaves the object in coincidence with itself. But the
result of these two operations is a rotation of $\pi$ about $Z$ (since
$X$ and $Y$ subtend an angle of $\pi/2$). Therefore, $Z$ is a 2-fold
rotation axis.
