\chapter{Symmetry Groups}

{\bf Comments on ``If any two of these elements belong to the symmetry
  group, then so does the third.''} (p.\ 37) Eqs.\ (2.6) and (2.6a)
show that each of the elements can be expressed as a combination of
the other two. Also all three commute with one another. So if two are
in the group, so must the third since we can produce the third as the
multiplication of the two.

{\bf Explanation of ``...if the $X$- and $Y$-axes are 2-fold rotation
  axes, then so it the $Z$-axis.''} (p.\ 38). $n$-fold rotation axis
means that the object is symmetric under rotation by $2\pi/n$ about
that axis (defined on p.\ 33).  A rotation of $\pi$ about $X$ leaves
the object in coincidence with itself. A following rotation of $\pi$
about $Y$ also leaves the object in coincidence with itself. But the
result of these two operations is a rotation of $\pi$ about $Z$ (since
$X$ and $Y$ subtend an angle of $\pi/2$). Therefore, $Z$ is a 2-fold
rotation axis.

{\bf Explanation of ``...if the line $OP$ is an $n$-fold axis, then we
  must also have $n$-fold axes along $OP^\prime...$''} (p.\ 38) If OP
is a $n$-fold rotation axis then if a $2\pi/n$ rotation about $OP^\prime$
does not bring the object into coincidence with itself, that means
that the distribution of points with respect to $OP^\prime$ is
different than that with respect to $OP$. That contradicts the
assumption that the vertical axis is a 3-fold symmetry axis.

{\bf ...all belong to the same class...} (p.\ 38) A $2\pi/n$ rotation about
$OP^\prime$ can be obtained by a $2\pi/3$ rotation bringing $OP^\prime$ into
coincidence with $OP$, followed by a rotation of $2\pi/n$ about $OP$,
followed by the reverse rotation of $2\pi/3$ back to $OP^\prime$. So the
rotation about $OP$ by $2\pi/n$ is conjugate to that about $OP^\prime$. In a
similar manner all $n$-fold rotations about $OP$, $OP^\prime$, and
$OP^{\prime\prime}$ are conjugate to one another. They thus form a
conjugate class.

{\bf Question: What is an example of an object that has multiple equivalent axes and rotation-reflection symmetry?} (p.\ 38)
