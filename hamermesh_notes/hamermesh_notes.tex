\documentclass{book}

\usepackage{amsmath}

\begin{document}

\chapter{Elements of Group Theory}

\section{Correspondences and transformations.}

{\bf p.\ 3}: A restatement with a cyclic permutation:
$$
\genfrac(){0pt}0{1234}{2341}
$$
means that the object in box 1 goes to box 2, the object in box 2 goes to box 3, {\it etc}. The numbers refer to the boxes and not the objects. In this notation the order of the columns is irrelevant, i.e.,
$$
\genfrac(){0pt}0{1324}{2431}
$$
is the same permutation.  If the objects were identified as A, B, C,
and D and start out in boxes 1, 2, 3, and 4 respectively,
$$
\genfrac{}{}{0pt}0{\tt 1234}{\tt ABCD}
$$
after application of the permutaion the box/object arrangement ends up as
$$
\genfrac{}{}{0pt}0{\tt 1234}{\tt DABC}
$$

\section{Groups. Definitions and examples.}

By a group of tranformations, $G$, we mean an aggregate of transformations of a given point set which
\begin{enumerate}
\item contains the identity transformation;
\item for every transformation $M$, also contains its inverse $M^{-1}$;
\item if it includes $M$ and $M^\prime$, also includes ther composite $MM^\prime$.
\end{enumerate}

[Rather than transformation, could use elements. Rather than composite, could use result of the group operation.]

{\bf p.\ 12}: (13)(12) should be intepreted as a transformation where first (12) is applied followed by (13). An example:
$$
\genfrac(){0pt}0{123}{231}
$$
is a cyclic permutation. If we start with box/objects
$$
\genfrac{}{}{0pt}0{\tt 123}{\tt ABC}
$$
then application of this permutation gives
$$
\genfrac{}{}{0pt}0{\tt 123}{\tt CAB}
$$
which is equivalant to applying the transposition (13), which gives
$$
\genfrac{}{}{0pt}0{\tt 123}{\tt CBA},
$$
followed by (23), which gives
$$
\genfrac{}{}{0pt}0{\tt 123}{\tt CAB},
$$
and is also equivalant to applying the transposition (12), which gives
$$
\genfrac{}{}{0pt}0{\tt 123}{\tt BAC},
$$
followed by (13), which gives
$$
\genfrac{}{}{0pt}0{\tt 123}{\tt CAB}.
$$
This is not the same as (13) followed by (12), which gives
$$
\genfrac{}{}{0pt}0{\tt 123}{\tt BCA}.
$$

A nice page with examples of cycles: https://math.stackexchange.com/questions/319979/how-to-write-permutations-as-product-of-disjoint-cycles-and-transpositions .

{\bf p.\ 14}: Call (a...xb...y) ``the cycle.'' Ignoring other possible independent cycles, we start with
$$
\genfrac{}{}{0pt}0{\tt 123456}{\tt ABCDEF},
$$
thus assuming $a=1$, $x=3$, $b=4$, and $y=6$. Applying the cycle we get
$$
\genfrac{}{}{0pt}0{\tt 123456}{\tt FABCDE}.
$$
Now apply (ab), aka (14). We get
$$
\genfrac{}{}{0pt}0{\tt 123456}{\tt CABFDE}.
$$
so this looks like (123)(456). More generally, the cycle puts $x \to
b$ and $y \to a$. So subsequent application of (ab) puts $x \to a$ and
$y \to b$ resulting in (a..x)(b...y)

\section{Subgroups. Cayley's theorem}

A subgroup is a subset of elements of a group that is a group itself. Two trivial subgroups: the indentity element and the entire group. These are improper subgroups, as opposed to proper subgroups.

The subgroup must use same group operation as the group.

\section{1.4 placeholder}

\section{1.5 placeholder}

\section{Invariant subgroups. Factor group. Homomorphism}

Starting from a subgroup $\mathcal H$ of group $G$ and any member of the group $a$, the set of elements $aha^{-1}$, where $h$ runs over all members of the subgroup $\mathcal H$, is a subgroup itself and is called the conjugate subgroup of $\mathcal H$ in $G$. If for all $a$, $a{\mathcal H} a^{-1}$ =  $\mathcal H$, we say that $\mathcal H$ is an invariant subgroup in $G$.

...isomorphism for group...a one-to-one correspondences was set up between elements $a$ of group $G$ and elements $a^\prime$ in a group $G^\prime$, such that $(ab)^\prime = a^\prime b^\prime$. By a homomorphic mapping of $G$ on $G^\prime$ we mean a similar correspondence which preserves prducts, but now several elements of $G$ may have the same image in $G^\prime$. [Example: cyclic group of order four mapped into cyclic group of order 2.]

\section{Direct Products}

A group $G$ is said to be a direct product of its subgroups ${\mathcal H_{\rm 1},...,H_{\rm n}}$ if:

\begin{enumerate}
\item The elements of different subgroups commute.
\item Every element of G is expressible in one and only one way as $g = h_1...h_n$, where $h_i$ is in ${\mathcal H}_i$, for $i = 1,...,n$
\end{enumerate}

\chapter{ch 2 placeholder}

\chapter{Group Representations}

\section{Linear vector spaces}

\def\bx{\mathbf x}
\def\by{\mathbf y}

We consider a set of objects ${\mathbf x,y,...}$ in which the elements can be ``multiplied'' by a complex number $\alpha$ or ``added'' to one another to give members of the same set. Such a set is called a {\it linear vector space} $L$:

If $\bx$ and $\by$ are in $L$, then
$$
\alpha{\mathbf x} \quad {\rm and} \quad {\bx+\by = \by+\bx}
$$
are also in $L$. The ``multiplication'' and ``addition'' must satisfy the conditions:
$$(\alpha + \beta)\bx = \alpha\bx + \beta\bx,$$
$$(\alpha\beta)\bx = \alpha(\beta\bx),$$
$$1\bx = \bx,$$
$$\alpha(\bx+\by) = \alpha\bx + \alpha\by.$$
The space $L$ will contain a zero vector (null vector), {\bf 0}, such that
$$\mathbf x + 0 = x \quad {\rm for all} \quad x.$$
Thus the linear vector space $L$ forms an abelian group under the ``addition'' operation, and its elements can be multiplied by complex numbers.

\section{}
\section{}
\section{Mappings; linear operators; matrix representatives; equivalence.}

Coordinates in unprimed system as function of those in primed system:
\begin{equation}
x_j = x^\prime_ia_{ij} = \tilde a_{ji}x^\prime_i\tag{3-16}
\end{equation}
\begin{equation}
y_i = T_i(x_1,\dots,x_n), \quad i=1,\dots,n\tag{3-25}
\end{equation}
\begin{equation}
x_i = T^{-1}_i(y_1,\dots,y_n), \quad i=1,\dots,n\tag{3-25a}
\end{equation}
\begin{equation}
J \equiv \left( \partial T_i\over\partial x_j \right) \tag{3-26}
\end{equation}
Not obvious why this a condition for the mapping to be one-to-one.

\section{Group representations}
\section{Equivalent representations; characters}

\end{document}
