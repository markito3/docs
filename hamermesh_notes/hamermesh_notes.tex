\documentclass{book}

\usepackage{amsmath}
\usepackage{hyperref}
\usepackage[marginparwidth=0pt]{geometry}

\begin{document}

\chapter{Elements of Group Theory}

\section{Correspondences and transformations.}

{\bf p.\ 3}: A restatement with a cyclic permutation:
$$
\genfrac(){0pt}0{1234}{2341}
$$
means that the object in box 1 goes to box 2, the object in box 2 goes
to box 3, {\it etc}. The numbers refer to the boxes and not the
objects. In this notation the order of the columns is irrelevant,
i.e.,
$$
\genfrac(){0pt}0{1324}{2431}
$$
is the same permutation.  If the objects were identified as A, B, C,
and D and start out in boxes 1, 2, 3, and 4 respectively,
$$
\genfrac{}{}{0pt}0{\tt 1234}{\tt ABCD}
$$
after application of the permutaion the box/object arrangement ends up as
$$
\genfrac{}{}{0pt}0{\tt 1234}{\tt DABC}
$$

p.\ 4: The degree of a permutation is simply the number of objects
being permuted.

\section{Groups. Definitions and examples.}

By a group of tranformations, $G$, we mean an aggregate of
transformations of a given point set which
\begin{enumerate}
\item contains the identity transformation;
\item for every transformation $M$, also contains its inverse $M^{-1}$;
\item if it includes $M$ and $M^\prime$, also includes ther composite $MM^\prime$.
\end{enumerate}

[Rather than transformation, could use elements. Rather than
  composite, could use result of the group operation.]

{\bf p.\ 7, Order of a Group}. The number of elements in a group is
called the {\it order} of the group.

{\bf Structure of a Group}. The structure of an abstract group is
determined by stating the result of the ``multiplication'' of every
ordered pair of elements, either by enumeration or by stating a
functional law, without any particularization of the nature of the
elements.

{\bf p.\ 8, Order of a group element}. If all powers of an element $a$
are distinct, then $a$ is of {\it infinite order}. If $n$ is the
smallest integer such that $a^n = e$ then $n$ is the order of the
element $a$.

{\bf Example of the order of a group element}. The cyclic group of order 4 is

\begin{equation*}
\begin{split}
e = (1234) \\
a = (2341) \\
b = a^2 = (3412) \\
c = a^3 = (4123)
\end{split}
\end{equation*}

The order of $e$ is 1 ($e^1 = e$). The order of $a$ is 4 ($a^4 = e$). The order of $b$ is 2 ($b^2 = e$). The order of $c$ is 4 since ($c^4 = e$).

{\bf p.\ 11}: The {\it four group} (a.k.a $V$) is

\begin{equation}
\begin{split}
  eabc \\
  aecb \\
  bcea \\
  cbae
\end{split}
\end{equation}
with the properties that $a^2 = b^2 = c^2 = e$, $ab = c$, $ac = b$,
and $bc = a$. Note that the four group has three elements (a, b, and
c) of order 2, and one element (e) of order 1. So none of the elements
of this group of order 4 is of order 4.

{\bf p.\ 12}. The permutations of degree $n$ form a group called the {\it symmetric group} of degree $n$, denoted by $S_n$.

{\bf p.\ 13}: (13)(12) should be intepreted as a transformation where
first (12) is applied followed by (13). An example:
$$
\genfrac(){0pt}0{123}{231}
$$
is a cyclic permutation. If we start with box/objects
$$
\genfrac{}{}{0pt}0{\tt 123}{\tt ABC}
$$
then application of this permutation gives
$$
\genfrac{}{}{0pt}0{\tt 123}{\tt CAB}
$$
which is equivalant to applying the transposition (13), which gives
$$
\genfrac{}{}{0pt}0{\tt 123}{\tt CBA},
$$
followed by (23), which gives
$$
\genfrac{}{}{0pt}0{\tt 123}{\tt CAB},
$$
and is also equivalant to applying the transposition (12), which gives
$$
\genfrac{}{}{0pt}0{\tt 123}{\tt BAC},
$$
followed by (13), which gives
$$
\genfrac{}{}{0pt}0{\tt 123}{\tt CAB}.
$$
This is not the same as (13) followed by (12), which gives
$$
\genfrac{}{}{0pt}0{\tt 123}{\tt BCA}.
$$

A nice page with examples of cycles:
https://math.stackexchange.com/questions/319979/how-to-write-permutations-as-product-of-disjoint-cycles-and-transpositions
.

{\bf p.\ 14}: Call (a...xb...y) ``the cycle.'' Ignoring other possible
independent cycles, we start with
$$
\genfrac{}{}{0pt}0{\tt 123456}{\tt ABCDEF}\quad,
$$
thus assuming $a=1$, $x=3$, $b=4$, and $y=6$. Applying the cycle we get
$$
\genfrac{}{}{0pt}0{\tt 123456}{\tt FABCDE}\quad.
$$
Now apply (ab), aka (14). We get
$$
\genfrac{}{}{0pt}0{\tt 123456}{\tt CABFDE}\quad.
$$
so this looks like (123)(456).

More generally, the cycle moves all objects to the next box, and in
particular the object in box $x$ moves to box $b$ and the object in
box $y$ moves to box $a$. So subsequent application of (ab) puts the
object that started in box $x$ into box $a$ and the object that
started in box $y$ into box $b$. The net effect is that objects in
boxes $a,...,x$ cycle among themselves. Likewise for the ojbect in
boxes $b,...,y$. I.e.,
$$
(ab)(a...xb...y) = (a...x)(b...y).
$$

\section{Subgroups. Cayley's theorem}

A subgroup is a subset of elements of a group that is a group
itself. Two trivial subgroups: the identity element and the entire
group. These are improper subgroups, as opposed to proper subgroups.

The subgroup must use same group operation as the group.

p. 15: An improper subgroup is the either the identity element alone or the whole group. Other subgroups are proper.

p. 15: Since the product of any pair of elements of $\mathcal H$ are
in $\mathcal H$ and each element of $\mathcal H$ contains its inverse,
for any element $a$ in $\mathcal H$, $aa^{-1} = e$ says that $e$ is an
element of $\mathcal H$.

p. 16: Notes on Cayley's Theorem: The symmetric group of degree $n$
has order $n!$. Cayley's Theorem refers to groups of order $n$ being
subgroups of the symmetric group of degree $n$ (not order $n$).

p.\ 17: Stated in the text: $ba_i = ba_j$ means $a_i = a_j$. The
result of both sides in the premise must be members of the group. The
statement is that they cannot result in the same member if $a_i$ and
$a_j$ are distinct members of the group. QUESTION: why? ANSWER: If
$ba_i = ba_j$ then $b^{-1}ba_i = b^{-1}ba_j$. Then $ea_i = ea_j$ and
so $a_i = a_j$. CONVOLUTED ANSWER:
\href{https://planetmath.org/uniquenessofinverseforgroups}{Inverses
  are unique}. Assume that $ba_i = ba_j$ and $a_i \neq a_j$. Then
$a^{-1}_i b^{-1} = a^{-1}_j b^{-1}$, and $a^{-1}_i b^{-1} b = a^{-1}_j
b^{-1} b$. So $a^{-1}_i = a^{-1}_j$ and since inverses are unique $a_i
= a_j$ which contradicts the assumption. So if $a_i \neq a_j$ then
$ba_i \neq ba_j$.

p. 19: A regular permutation is one for which no elements remain in
their original order. No definition of a regular subgroup is given. My
take: a regular subgroup is one where all elements are regular
permutations excluding the identity.

p. 19: Alternate statement of a property of certain
subgroups:\hfil\break Groups with only regular permutations (other
than the identity) have the following property. For any given box,
and for a given element (i.e., permutation) of the group, the image
box under the permutation is distinct from that of all other elements
of the group.

p. 20: Group $G$ is {\it cyclic} if there exists $a$, a member of $G$,
such that the cyclic subgroup generated by $a$ ($a,a^2,a^3,...$),
equals all of $g$

p.21: With reference to a proper subgroup $H$ of order $m$, of a
finite group $G$ of order $n$, where $n/m = h$, $h$ is called the
index of the subgroup. The set of members of the $G$ formed by
multiplying by $a$ on the left of each memeber of $H$, where $a$ is a
member of $G$ and not a member of $H$, is called a left coset of
$H$. Multiplying on the right gives a right coset. Note that cosets
cannot be subgroups of $G$ since they do not contain the identity.

\section{Cosets. Lagrange's theorem.}

{\bf Lagrange's Theorem}. The order of a subgroup of a finite group is
a divisor of the order of the group.

{\bf p.\ 22}: If the order of an element $a$ of a group $G$ is $n$
then $a^n = e$. By definition of the order of an element,
$a,a^2,...,a^n$ are all distinct. So these powers of $a$ form a group
of order $n$. I.e., if $a$ is an element of order $n$, it is a member
of a subgroup of $G$ of order $n$.

The cyclic groups are \href{https://en.wikipedia.org/wiki/Cyclic_group}{defined on Wikipedia}. From that article:

\begin{quote}
Every finite cyclic group of order $n$ is isomorphic to the additive group of $Z/nZ$, the integers modulo n.
\end{quote}

% group of order 6 discussion

Lagrange's theorem can be used for finding the possible structures of
groups of a given order. As an example we find all structures of
order 6. Since the order of the group is six the order of each of its
elements is a divisor of 6 i.e., 1, 2, 3 or 6.
\begin{quote}
If an element $a$ is of order $n$ then the group elements $a,a^2,...,a^n$ are distinct and form a subgroup of order $n$. By Lagrange's theorem, $n$ must be a divisor of 6.
\end{quote}
If the group contains an element
$a$ of order 6 then the group is the cyclic group $a,a^2,...,a^6 = e$.
\begin{quote}
This subgroup has the same structure as the cyclic group, so it is in
fact the cyclic group.
\end{quote}
To find other possible structures, we suppose that the group
contains no element of order 6, but has an element $a$ of order
3. Thus the group contains the subgroup $a,a^2,a^3 = e$.
\begin{quote}
  $a$ cannot be $e$ since $a^2 = e$ and $a$ is a member of a group of
  order 1. $a^2$ cannot be $e$ otherwise $aa^2 = a^3$ would be equal
  to $a$ and thus $a$ and $a^3$ would not be distinct.
\end{quote}
If the group also contains another element, $b$, then it contains the
six distinct elements $e,a,a^2,b,ba,ba^2$.
\begin{quote}
  $b$ is distinct from $e,a,a^2$ by assumption. $ba$ is distinct from
  $b$ since $a$ is not the identity and is distinct from $a$ since $b$
  is not the identity. $ba$ is distinct from $e$ because that would
  imply that $b = a^{-1}$ but $a^{-1} = a^2$ and $b$ is distinct
  from $a^2$ by assumption. $ba$ is distinct from $a^2$ since that
  would imply that $b = a$, which is false by assumption. Similar
  arguments for the $ba^2$ being distinct from the other five
  elements.
\end{quote}
The element
$b$ has order two or three.
\begin{quote}
  It cannot be of order one since it is distinct from the identity by
  assumption. It cannot be order six since we have assumed that there
  is no element of order 6.
\end{quote}
If the order of $b$ is 3, that is, $b^3 = e$,
the element $b^2$ must be one of the six elements listed.
We cannot have $b^2 = e$ (since we assumed that $b$ is of
order 3), and $b^2 = b, ba$, or $ba^2$ implies $b
= e, a$, or $a^2$, respectively, which contradicts our
assumption that $b$ is distinct from these elements. Furthermore
$b^2 = a$ implies $ba = e$, and $b^2 = a^2$ implies $ba = e$,
both of which contradict our assumptions. Thus the order of $b$
cannot be 3 and we must have $b^2 = e$. The product $ab$ cannot
be $e, a, a^2$, or $b$. If $ab = ba$ then
\begin{multline}
(ab)^2 = (ab)(ab) = (ab)ba = ab^2a = a^2, \\
(ab)^3 = a^2(ab) = b; (ab)^4 = a, (ab)^5 = ba^2, (ab)^6 = e,
\end{multline}
and therefore the group would contain an element $ab$ of order 6
contrary to assumption. (This can be done more briefly: $a$ is of order
2, $b$ of order 3, and $ab = ba$, that is, $a$ and $b$ commute. Taking powers
we see that the order of $ab$ is the least common multiple of the
orders of $a$ and $b$.) We are now left with
$$
b^2 = e, ab = ba^2;
$$
$ab = ba^2$ implies $(ab)^2 = (ab)ba^2 = e$.

This last assumption leads to no contradictions.

\section{Conjugate classes.}

\section{Invariant subgroups. Factor group. Homomorphism}

Starting from a subgroup $\mathcal H$ of group $G$ and any member of
the group $a$, the set of elements $aha^{-1}$, where $h$ runs over all
members of the subgroup $\mathcal H$, is a subgroup itself and is
called the conjugate subgroup of $\mathcal H$ in $G$. If for all $a$,
$a{\mathcal H} a^{-1}$ = $\mathcal H$, we say that $\mathcal H$ is an
invariant subgroup in $G$.

...isomorphism for group...a one-to-one correspondences was set up
between elements $a$ of group $G$ and elements $a^\prime$ in a group
$G^\prime$, such that $(ab)^\prime = a^\prime b^\prime$. By a
homomorphic mapping of $G$ on $G^\prime$ we mean a similar
correspondence which preserves prducts, but now several elements of
$G$ may have the same image in $G^\prime$. [Example: cyclic group of
  order four mapped into cyclic group of order 2.]

\section{Direct Products}

A group $G$ is said to be a direct product of its subgroups ${\mathcal H_{\rm 1},...,H_{\rm n}}$ if:

\begin{enumerate}
\item The elements of different subgroups commute.
\item Every element of G is expressible in one and only one way as
  $g = h_1...h_n$, where $h_i$ is in ${\mathcal H}_i$, for $i = 1,...,n$
\end{enumerate}

\chapter{ch 2 placeholder}

\chapter{Group Representations}

\section{Linear vector spaces}

\def\bx{\mathbf x}
\def\by{\mathbf y}

We consider a set of objects ${\mathbf x,y,...}$ in which the elements
can be ``multiplied'' by a complex number $\alpha$ or ``added'' to one
another to give members of the same set. Such a set is called a {\it
  linear vector space} $L$:

If $\bx$ and $\by$ are in $L$, then
$$
\alpha{\mathbf x} \quad {\rm and} \quad {\bx+\by = \by+\bx}
$$
are also in $L$. The ``multiplication'' and ``addition'' must satisfy the conditions:
$$(\alpha + \beta)\bx = \alpha\bx + \beta\bx,$$
$$(\alpha\beta)\bx = \alpha(\beta\bx),$$
$$1\bx = \bx,$$
$$\alpha(\bx+\by) = \alpha\bx + \alpha\by.$$
The space $L$ will contain a zero vector (null vector), {\bf 0}, such that
$$\mathbf x + 0 = x \quad {\rm for all} \quad x.$$
Thus the linear vector space $L$ forms an abelian group under the
``addition'' operation, and its elements can be multiplied by complex numbers.

\section{}
\section{}
\section{Mappings; linear operators; matrix representatives; equivalence.}
\section{Group representations}
\section{Equivalent representations; characters}

\end{document}
