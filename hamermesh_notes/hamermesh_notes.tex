\documentclass{book}

\usepackage{amsmath}

\begin{document}

\chapter{Elements of Group Theory}

\section{}
\section{}

\section{1-3 Subgroups. Cayley's theorem}

A subgroup is a subset of elements of a group that is a group itself. Two trivial subgroups: the indentity element and the entire group. These are improper subgroups, as opposed to proper subgroups.

The subgroup must use same group operation as the group.

\section{}
\section{}

\section{1-6 Invariant subgroups. Factor group. Homomorphism}

Starting from a subgroup $\mathcal H$ of group $G$ and any member of the group $a$, the set of elements $aha^{-1}$, where $h$ runs over all members of the subgroup $\mathcal H$, is a subgroup itself and is called the conjugate subgroup of $\mathcal H$ in $G$. If for all $a$, $a{\mathcal H} a^{-1}$ =  $\mathcal H$, we say that $\mathcal H$ is an invariant subgroup in $G$.

...isomorphism for group...a one-to-one correspondences was set up between elements $a$ of group $G$ and elements $a^\prime$ in a group $G^\prime$, such that $(ab)^\prime = a^\prime b^\prime$. By a homomorphic mapping of $G$ on $G^\prime$ we mean a similar correspondence which preserves prducts, but now several elements of $G$ may have the same image in $G^\prime$. [Example: cyclic group of order four mapped into cyclic group of order 2.]

\section{1-7 Direct Products}

A group $G$ is said to be a direct product of its subgroups ${\mathcal H_{\rm 1},...,H_{\rm n}}$ if:

\begin{enumerate}
\item The elements of different subgroups commute.
\item Every element of G is expressible in one and only one way as $g = h_1...h_n$, where $h_i$ is in ${\mathcal H}_i$, for $i = 1,...,n$
\end{enumerate}

\chapter{}

\chapter{Group Representations}

\section{3-1 Linear vector spaces}

\def\bx{\mathbf x}
\def\by{\mathbf y}

We consider a set of objects ${\mathbf x,y,...}$ in which the elements can be ``multiplied'' by a complex number $\alpha$ or ``added'' to one another to give members of the same set. Such a set is called a {\it linear vector space} $L$:

If $\bx$ and $\by$ are in $L$, then
$$
\alpha{\mathbf x} \quad {\rm and} \quad {\bx+\by = \by+\bx}
$$
are also in $L$. The ``multiplication'' and ``addition'' must satisfy the conditions:
$$(\alpha + \beta)\bx = \alpha\bx + \beta\bx,$$
$$(\alpha\beta)\bx = \alpha(\beta\bx),$$
$$1\bx = \bx,$$
$$\alpha(\bx+\by) = \alpha\bx + \alpha\by.$$
The space $L$ will contain a zero vector (null vector), {\bf 0}, such that
$$\mathbf x + 0 = x \quad {\rm for all} \quad x.$$
Thus the linear vector space $L$ forms an abelian group under the ``addition'' operation, and its elements can be multiplied by complex numbers.

\section{}
\section{}
\section{3-4}

Coordinates in unprimed system as function of those in primed system:
\begin{equation}
x_j = x^\prime_ia_{ij} = \tilde a_{ji}x^\prime_i\tag{3-16}
\end{equation}
\begin{equation}
y_i = T_i(x_1,\dots,x_n), \quad i=1,\dots,n\tag{3-25}
\end{equation}
\begin{equation}
x_i = T^{-1}_i(y_1,\dots,y_n), \quad i=1,\dots,n\tag{3-25a}
\end{equation}
\begin{equation}
J \equiv \left( \partial T_i\over\partial x_j \right) \tag{3-26}
\end{equation}
Not obvious why this a condition for the mapping to be one-to-one.

\end{document}
