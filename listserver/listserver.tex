\documentclass{article}

\begin{document}

\section{History}

The standard Mailman software at JLab has worked well for us and does
meet most of our needs. But a few years ago, we wanted to start a
software help forum and we thought that the mailman interface was too
hard to search. For that application you want to be able to see if
your question has been asked and answered before, \` a la StackOverflow. We looked at a
JLab-managed Drupal site using the forum module, but that had problems
handling pure email sending and receiving. We settled on Google
Groups. It had great search, web-based reading and posting, and
email-based sending and receiving. You could use your JLab email
account for all transactions except posting on the website. You could
easily dis-allow posting by non-members, and invite or add new
members. Problem solved, we thought.

On another front, we had had our collaborator Matt Shepherd
maintaining a Hypernews site at Indiana University. It was used for
private groups, to discuss papers in preparation for publication. It
had all the right features for us, but the system admin overhead was a
an issue, especially setting up new topics. He asked about
transitioning this function to Googlegroups. Note that for the
aforementioned software help list, we wanted to have all messages
publicly readable on the Google Groups site. For these applications,
the archive needed to be private. Fine. A private archive is no
problem on Google Groups. The problem was that to read the archive on
the web, you had to authenticate to Google, and so you needed to use a
Google account. Some people objected to using their private Google
account for work related business, plus there were objections on
privacy grounds. We had a work around where we told people about the
administrative google account (group owner account), including the
password. This would allow reading, but this had its own problems,
some obvious and some to be mentioned below. Also we started to notice
conflicts in privilege settings with at universities using GSuite that
would require action by the university's GSuite administrator to
solve.

Lately, more of us would like to administer a private group, but that
has turned out to be an expert-only operation. So we are looking for
other solutions.

\section{Requirements}

In this section while discussing requirements, we will refer to our
experiences to illustrate requirement-related strengths and weakness
we have seen. We will refer the the Mailman system as MM and
GoogleGroups as GG in what follows.




\section{Options}

\end{document}

