\documentclass{article}

\begin{document}

\begin{center}
  {\Large Requirements for a Email List Server for JLab} \\
  \medskip
  {\large Mark Ito \\
    June 4, 2020} \\
  GlueX-doc 4522-v3
\end{center}

\section{History}

The standard Mailman software at JLab has worked well for us and does
meet most of our needs. But a few years ago, we wanted to start a
software help forum and we thought that the mailman interface was too
hard to search. For that application you want to be able to see if
your question has been asked and answered before, \` a la StackOverflow. We looked at a
JLab-managed Drupal site using the forum module, but that had problems
handling pure email sending and receiving. We settled on Google
Groups. It had great search, web-based reading and posting, and
email-based sending and receiving. You could use your JLab email
account for all transactions except posting on the website. You could
easily dis-allow posting by non-members, and invite or add new
members. Problem solved, we thought.

On another front, we had had our collaborator Matt Shepherd
maintaining a HyperNews site at Indiana University. It was used for
private groups, to discuss papers in preparation for publication. It
had all the right features for us, but the system admin overhead was
an issue, especially setting up new topics. He asked about
transitioning this function to GoogleGroups. Note that for the
aforementioned software help list, we wanted to have all messages
publicly readable on the Google Groups site. For these paper discussions,
the archive needed to be private. Fine. A private archive is no
problem on Google Groups. The problem we did run into was that to read
the archive on
the web, you had to authenticate to Google, therefore you needed to use a
Google account. Some people objected to using their private Google
account for work related business, or objected to having to create yet
another account, plus there were objections on
Google-specific privacy grounds.
We had a work around where we told people about the
group owner account Google account, including the
password. This would allow reading, but this had its own problems,
some obvious and some to be mentioned below. Also we started to notice
conflicts in privilege settings with at universities using GSuite that
would require action by the university's GSuite administrator to
solve.

Lately, more of us would like to administer a private Google group, but that
has turned out to be an semi-expert-only operation.Group administration has
a not insignificant learning curve. So we are looking for
other solutions.

\section{Requirements}

In this section while discussing requirements, we will refer to our
experiences to illustrate requirement-related strengths and weakness
we have seen. We will refer the the Mailman system as MM and
GoogleGroups as GG in what follows.

\subsection{Email Access}

Almost goes without saying, but users need to have the ability to
receive messages and post messages using email and email only. This is
the tradition in the field.

See the comment about Drupal in the history section above.

\subsection{Archive}

All products have an archive. Should be able to choose between public
(readable by anyone) and private (authentication needed to read)
archives.

\subsection{Web Access}

Not only should the archives be accessible via a web browser (which
all products support), but there should be an ability to post new
messages or reply to old ones via the web.

\subsection{Subscriber Community}

People from outside JLab need to be able to participate fully as
subscribers and preferably as managers. One can assume subscribers
have JLab CUE accounts for functions where authentication is
necessary.

\subsection{Search/Browse}

Need facility so that

\begin{itemize}
  
\item past conversions only vaguely remembered can be found.

\item folks with questions can search to see if the question has
  already been asked and answered.

\item browsing through the history chronologically is supported.

\end{itemize}

GG is good for this, MM very poor.

\subsection{Agile Deployment}

When users want a new email list, they should be able to create one
on-demand (with appropriate limits on creators) rather than waiting
for CNI to act on a request. Same for deleting a list that was a bad
idea.

GG is good for this, MM not.

\subsection{Group Management}

Management of the email list should be do-able by a team of users, not
a single individual. Here we mean subscription approval, removal of
members, visibility settings, etc. This avoids problems with
vacations, travel, and job changes of managers.

\subsection{Authentication}

It would be ideal if JLab credentials could be used to authenticate in
cases where authentication is necessary. This way a separate
access-control management task is finessed.

\subsection{File Attachments}

It should be possible to add attachments to posted messages. The big
use case here is attachment of PDF files. Cyber security is a concern
here.

\subsection{System Administration}

Being users, we would like CNI to be able to handle system
configuration, backups, and upgrades.

\section{Product List}

At this writing this list is pretty minimal. Other ideas should
certainly need to be explored. For completeness I listed the products
we are trying to get away from.

\begin{description}

\item{\bf Mailman} See the discussion above.

\item{\bf Google Groups} See the discussion above.

\item{\bf HyperNews} See the discussion above. This is old and probably does not have active support.

\item{\bf Drupal Forum} See the discussion above.

\item{\bf Discourse} ``Civilized discussion for your community''
{\tt https://www.discourse.org/} Hall B is already using the free version. The
  Geant4 Collaboration has moved to Discourse after SLAC took down
  their HyperNews site for security reasons.

  \item{\bf LISTSERV} ``Integrated Email List, Web and Database Communication'', {\tt http://www.lsoft.com/products/listserv.asp}

\item{\bf groups.io} ``Email Groups. Supercharged'' {\tt https://groups.io/}

\item{\bf Microsoft Teams}

\item{\bf Slack} No email interface of which I am aware. Probably not a real candidate for that reason.

\end{description}

\end{document}

