\documentclass{book}

\usepackage{amsmath}
\usepackage{hyperref}

\begin{document}

\chapter{Quarks and leptons}

\section{Preamble}

\subsection{Why high energies?}

References:

\url{https://en.wikipedia.org/wiki/Angular\_aperture}
\url{https://www.leica-microsystems.com/science-lab/life-science/microscope-resolution-concepts-factors-and-calculation/}

The de Broglie relation:
$$
E = mv^2
$$
where $E$ is the energy and $v$ is the velocity of a particle in analogy with $E=mc^2$. Also
$$
E = h\nu
$$
where $h$ is Plank's constant and $\nu$ is the particle's frequency. And also
$$
v = \lambda\nu
$$
the wave velocity where $\lambda$ is the wavelength. Setting the two formulae for energy equal gives
$$
\lambda = {h \over mv} = {h \over p}
$$
by indentifying momentum $p$ with $mv$.

\section{The Standard Model of particle physics}
\section{Particle classification: fermions and bosons}
\section{Particles and antiparticles}
\section{Free particle wave equations}
Eq.\ 1.15 comes from substituting the energy and momentum operators directly into the equation $E = p^2/(2m)$.

Hard to understand from first principles.

\section{Helicity states: helicity conservation}

Hard to understand from first principles.

\section{Lepton flavours}

p. 21: Cannot corroborate the reason that a ``massless neutrino cannot possess a magnetic dipole moment.''

\section{Quark flavours}
\section{The cosmic connection}

\end{document}
