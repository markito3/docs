
\documentstyle[11pt,apsabinv]{article}
\nofiles
\MeetingID{SES19}
%\DateSubmitted{20190930}
\LogNumber{SES19-2019-000162}
\SubmittingMemberSurname{Ito}
\SubmittingMemberGivenName{Mark}
%\SubmittingMemberID{USA}
\SubmittingMemberEmail{marki@jlab.org}
\SubmittingMemberAffil{Jefferson 
Lab}
\PresentationType{}
\SortCategory{}{}{}{}
\received{30 Sep 
2019}
\begin{document}
\Title{Plans for a Measurement of Charged and Neutral 
Pion Polarizabilities with 
GlueX}
\AuthorSurname{Ito}
\AuthorGivenName{Mark}
%\AuthorEmail{marki@jlab.org}
\AuthorAffil{Jefferson 
Lab}
\begin{abstract}
We report on the status of preparations for a measurement 
of the
charged pion polarizability. The experiment is approved to run in Hall
D at Jefferson using the GlueX detector. In addition we report on
plans for a proposal to measure the neutral pion polarizability, also
with GlueX, to run simultaneously with the charged pion experiment.
For both charged and neutral pions the polarizabilities are fully
predicted at leading order in quark masses, and thus represent a
sensitive test of chiral dynamics. Polarizabilities are accessed
through a measurement of the absolute cross section of $\pi^+\pi^-$
and $\pi^0\pi^0$ production near threshold via the Primakoff effect,
$\gamma \gamma^*\rightarrow \pi\pi$ where the virtual photon is
provided by the Coulomb field of the target nucleus. Construction of a
new detector system for identification of muon pair production, which
present a background for the charged mode, are underway and will be
described.
\end{abstract}
\end{document}

